\section{Recursos Materiais}

\textbf{Material bibliográfico.} O material bibliográfico utilizado é composto principalmente por periódicos científicos, artigos e livros. Este material estará disponível para o autor \recom{do}{desse} trabalho das seguintes maneiras: fisicamente, via Biblioteca Central da UFES, e eletronicamente, via rede de Internet da UFES, permitindo o acesso ao acervo eletrônico próprio da universidade e ao acervo cujo acesso tenha sido adquirido pela universidade.

\section{Recursos Computacionais}

\textbf{Recursos de Software}
%\textbf{1. \txr{Descrição das ferramentas de programação usadas}}

Na etapa de criação e configuração da plataforma de orquestração de contêineres pretende-se utilizar o software Kubernetes, \recom{. Alem disso,}{além de} todo o conjunto de softwares, tais como RabbitMQ e Zipkin, que compõem a plataforma do PIS. Já durante as etapas de adaptação das aplicações do sistema, desenvolvimento e validação do controlador proposto será utilizada a linguagem de programação em Python e o conjunto de ferramentas de suporte à aplicação do PIS previamente desenvolvido. Além disso, todas aplicações devem ser empacotadas na forma de contêineres utilizando o software Docker.


\textbf{Recursos de Hardware}
%\textbf{2. \txr{Descrição do hardware usado}}

Como dispositivo IoT, pretende-se utilizar uma Raspberry model B com as seguintes configurações: (i) sistema operacional Linux, distribuição Ubuntu Server 22.04; (ii) \recom{processor}{processador} Cortex-A72 (ARM v8) 64-bit SoC, 1.5GHz com 4 núcleos físicos; (iii) memória SDRAM de 4GB LPDDR4 3200MHz; (iv) conexão Wi-Fi de 2.4 GHz e 5.0 GHz IEEE 802.11ac; (v) cartão de memória de 32GB (classe 10, até 80 MB/s); (v) Raspberry Pi Camera Module 2. 

Além disso, pretende-se utilizar um \textit{datacenter} presente no laboratório para realização dos testes de validação do sistema proposto. Neste \textit{datacenter}, a máquina com as seguintes configurações será utilizada como máquina de computação de nuvem: (i) sistema operacional Linux, distribuição Ubuntu Server 22.04; (ii) \recom{processor}{processador} Intel Xeon, 3.2GHz com 28 núcleos físicos; (iii) memória DDR4 de 64GB; (iv) conexão ethernet de 1GHz; (v) SSD de 480GB; (iv) 3 NVIDIA GeForce RTX 3060 e 1 NVIDIA GeForce RTX 3090.
